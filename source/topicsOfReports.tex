%!TeX TS-program = xelatex % | mv %.pdf ./source/.
\documentclass[14pt,a4paper]{article}
\usepackage{cmap} %Улучшает поиск по pdf документу
\usepackage[pdftex,
    pdfauthor={К.С.~Пилипенко},
    pdftitle={Список тем для докладов студентов по радиологии},
    pdfsubject={The Subject},
    pdfkeywords={Первое ключевое слово, второе ключевое слово},
    pdfproducer={LuaLatex with hyperref},
    pdfcreator={Lualatex},
    hidelinks
]{hyperref}
%%%%%%%%%%%%%%Пользовательские команды%%%%%%%%%
\usepackage{latexsym,amsmath,amssymb,amsbsy,graphicx}
\usepackage{icomma}
\usepackage[version=4]{mhchem} % the canonical chemistry package (example: \ce{^{32}_{15}P})
\usepackage{graphicx}
\graphicspath{{images/}}
\DeclareGraphicsExtensions{.pdf,.png,.jpg}
%%%%%%%%%%%%%%%%%%%%%%%%Оформление по ГОСТУ
\usepackage{fontspec}
\setmainfont[Renderer=Basic,Ligatures={TeX}]{Times New Roman}
\usepackage[english,russian]{babel} %Поддержка русской локализации
\usepackage[14pt]{extsizes} % для того чтобы задать нестандартный 14-ый размер шрифта
\usepackage{indentfirst} %Задаёт отступ самого первого абзаца
\setlength\parindent{1.25cm}
\usepackage[a4paper, left=3cm, top=1.5cm, right=1.5cm, bottom=2cm]{geometry}
\usepackage{setspace}
\sloppy %Выравнивание текст по ширине и решение проблемы переполнением строки
\onehalfspacing %Полуторный интервал
\usepackage{mathtext} % русские буквы в формулах
\usepackage{caption} %заголовки плавающих объектов
\captionsetup[figure]{name=Рис.} % меняет название рисунков на русское
%%%%%%%%%%%%%%%%%%%%%%%%%%%%%
\author{\href{mailto:www-kirill.pilipenko@yandex.ru}{К.С.~Пилипенко}} %Через \and можно добавить ещё авторов
\date{\selectlanguage{russian}\today}
\begin{document}
\begin{center}
    \textbf{\Large{Темы докладов по радиологии}}
\end{center}
    \begin{enumerate}
        \item История развития лучевой диагностики и терапии. Открытие КТ, МРТ, ПЭР, УЗ-терапии и других методов.
        \item Рентгенологические методы. Как влияют параметры рентгеновской трубки на получение рентгеновского изображения. Искусственное контрастирование органов. Рентгенография. Рентгеноскопия. Флюорография. Томография. Компьютерная томография. Ангиография.
        \item Интервенционная радиология
        \item Медицинские линейные ускорители (бетатрон, микротрон, линейный ускоритель электронов, циклотрон, синхрофазотрон). Принцип действия, различия, применение.
        \item Физические основы лучевой терапии фотонами, электронами, протонами, ионами и нейтронами.
        \item Оборудование классической дистанционной лучевой терапии. Гамма-установки с радиоактивным источником.  Томотерапия. Гамма-нож. Кибернож.
        \item Оборудование контактной лучевой терапии. Аппараты брахитерапии. Аппараты интраоперационной лучевой терапии
        \item Радиационные синдромы: костномозговой, желудочно-кишечный, церебральный. 
        \item Медицинское изображение как объект информатики. Аналоговое, цифровое, аналогово-цифровое изображение. Сферы использования, преимущества, недостатки.
        \item Лучевая диагностика костно-суставной системы. Нормальная анатомия. Семиотика болезней костей. Переломы и вывихи. Артриты и артрозы. Опухоли костей.
        \item Лучевое исследование функции легких. Общие показатели нормального легочного и корневого рисунка. Патологические образования на лучевых снимках.
    \end{enumerate}
\end{document}